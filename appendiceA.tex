\chapter{Tabella degli integrali}
Data la funzione $f_{(n)}$ definita come:
\begin{equation}
	f_{(n)} = \int_0^{\infty} n^x \cdot e^{-\alpha \cdot x^2} d x
	\label{eq:AppA:IntegraleGenerico}
\end{equation}
con $\alpha > 0$ abbiamo:
\begin{table*}[h]
	\centering
		\begin{tabular}{lr}
			n		&		$f_{(n)}$ \\ \hline
			0		&		$\frac{1}{2} \sqrt{\frac{\pi}{\alpha}} $  			\vspace{0.2cm} \\ \hline
			1		&		$\frac{1}{2 \alpha}$ 														\vspace{0.2cm} \\ \hline
			2		&		$\frac{1}{4} \sqrt{\frac{\pi}{\alpha^3}} $ 			\vspace{0.2cm} \\ \hline
			3		&		$\frac{1}{2 \alpha^2}$ 													\vspace{0.2cm} \\ \hline
			4		&		$\frac{3}{8} \sqrt{\frac{\pi}{\alpha?5}} $ 			\vspace{0.2cm} \\ \hline
			5		&		$\frac{1}{\alpha^3}$ 														\vspace{0.2cm} \\ \hline
			6		&		$\frac{15}{16} \sqrt{\frac{\pi}{\alpha^7}} $ 		\vspace{0.2cm} \\ \hline
			7		&		$\frac{3}{\alpha^4}$ 														\vspace{0.2cm} \\ \hline
		\end{tabular}
\end{table*}

Si noti, anche, che per $n$ pari vale la seguente regola:
\begin{equation}
	\int_{-\infty}^{\infty} n^x \cdot e^{-\alpha \cdot x^2} d x = 2 \cdot f_{(n)}
	\label{eq:AppA:IntegraleGenericoPari}
\end{equation}
mentre per per $n$ dispari:
\begin{equation}
	\int_{-\infty}^{\infty} n^x \cdot e^{-\alpha \cdot x^2} d x = 0
	\label{eq:AppA:IntegraleGenericoDispari}
\end{equation}